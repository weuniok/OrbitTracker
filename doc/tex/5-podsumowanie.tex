\clearpage % Rozdziały zaczynamy od nowej strony.
\section{Podsumowanie}

W ramach przedstawionego zagadnienia sprawdzono stosowalność metody Gaussa do wstępnego wyznaczania orbit obiektów kosmicznych na podstawie obserwacji naziemnych. W tym celu zaimplementowano metodę w formie programu oraz przygotowano zbiory danych walidacyjnych. Metoda wykorzystuje założenie, że czasy między kolejnymi pomiarami są niewielkie względem okresu orbity i dla przypadków zgodnych z założeniami jest skuteczna. Jeśli międzyczasy są zbyt małe, to nawet drobne błędy w obserwacji mogą spowodować duże rozbieżności w obliczonej orbicie. Wykonanie pomiarów w trakcie jednego przebiegu orbity - podczas wschodu, zenitu i zachodu obiektu pozwala uzyskać optymalne wyniki, jednocześnie zachowując pewność, że obserwowany jest ten sam obiekt.