\clearpage % Rozdziały zaczynamy od nowej strony.
\section{Cel i zakres projektu}

Celem pracy było wykorzystanie wybranej metody wyznaczania orbity statku kosmicznego na podstawie obserwacji naziemnych. Zaimplementowano metodę Gaussa wraz z poprawką iteracyjną. Do sprawdzenia poprawności działania metody stworzono generator obserwacji, który na podstawie modelu keplerowskiego i założonych elementów orbitalnych wyznacza dane obserwacyjne. Wykonano również walidację korzystając z danych wygenerowanych przy pomocy zewnętrznego programu do symulacji nocnego nieba Stellarium \cite{Stellarium}.
